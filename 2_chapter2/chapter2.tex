%!TEX root = ../thesis.tex
%*******************************************************************************
%****************************** Second Chapter *********************************
%*******************************************************************************
% Use [] to create a short form of the title which is used in the header of the chapter when the title is too long for the running headers
\chapter[Literature Review]{Literature Review}

\ifpdf
    \graphicspath{{Chapter2/Figs/Raster/}{Chapter2/Figs/PDF/}{Chapter2/Figs/}}
\else
    \graphicspath{{Chapter2/Figs/Vector/}{Chapter2/Figs/}}
\fi

\section{Speech Technology and Automatic Speech Recognition}
%\label{chapter2:Speech_technology_and_automatic_speech_recognition}

\subsection{Speech Representations}
%\label{chapter2:limitations_of_conventional_approaches}
% FFT, Spectrogram, Mel-Cepstrum-Coefficients (and then CNN feature extraction?)

\subsection{Deep Neural Nets}

\subsection{Feature Extraction Via CNNs}
%First feature extractors, RESNET etc.

\subsection{DNNs in ASR}
%First uses of DNNs for ASR

\subsection{CNNs in Speech}

\section{Language models}
%\label{chapter2:perceptual_criteria_vqa}

\subsection{Classic Language Modelling}
\label{ssec:vmaf_non_differentiable}
%Beam search

\subsection{Transformers} 
%Deep learning
%LSTM and Transformers

\section{Self-supervised learning}
\label{litreview:nn_artifact}

\subsection{Large-Scale ASR Training}

%\subsection{Neural based Approaches for Artifact Suppression}

\section{Data Bias in Speech}

\subsection{Low-resource and Zero-shot Learning}

\subsection{Dominance of English Data in Speech}

\section{Model Evaluation}
%\label{chapter2:training_NN_percept_crit}

\subsection{Evaluation Through Inference}

\subsection{Evaluation Through Weight Activation}

\subsection{Evaluation Through Model Compression}

\section{Open Questions}

\section{Datasets}

\section{Conclusion}

% Put actual numbers in papers 